\documentclass[12pt,a4paper]{article}
\usepackage[utf8]{inputenc}
\usepackage[T1]{fontenc}
\usepackage{geometry}
\usepackage{hyperref}
\usepackage{listings}
\usepackage{xcolor}
\usepackage{booktabs}

\geometry{margin=2.5cm}

\definecolor{codegreen}{rgb}{0,0.6,0}
\definecolor{backcolour}{rgb}{0.95,0.95,0.92}

\lstdefinestyle{mystyle}{
    backgroundcolor=\color{backcolour},
    commentstyle=\color{codegreen},
    keywordstyle=\color{magenta},
    basicstyle=\ttfamily\footnotesize,
    breaklines=true,
    numbers=left,
    numbersep=5pt,
    tabsize=2
}
\lstset{style=mystyle}

\title{
    \textbf{Amal Mobile Technical Report} \\
    \large React Native Application Architecture
}
\author{Samir Guenchi}
\date{December 2024}

\begin{document}

\maketitle

\begin{abstract}
This report presents the technical decisions for the Amal mobile application, a cross-platform app providing drug addiction support for users in Algeria. The application uses React Native with Expo, enabling deployment to both Android and iOS from a single codebase. This document justifies the framework selections and architectural decisions.
\end{abstract}

\tableofcontents
\newpage

\section{Introduction}

The Amal mobile application requirements include:
\begin{itemize}
    \item Cross-platform deployment (Android and iOS)
    \item Multilingual support with RTL layouts
    \item Real-time AI chat interface
    \item Offline capability for basic features
    \item Professional design for ages 20-60
\end{itemize}

\section{Framework Selection}

\subsection{React Native}

React Native was selected for cross-platform mobile development:

\begin{enumerate}
    \item \textbf{Code Sharing}: Single codebase for Android and iOS reduces development time by 40-50\%.
    
    \item \textbf{Native Performance}: Unlike hybrid frameworks, React Native compiles to native components.
    
    \item \textbf{React Ecosystem}: Leverages existing React knowledge and libraries from the web frontend.
    
    \item \textbf{Hot Reloading}: Instant feedback during development accelerates iteration.
    
    \item \textbf{Large Community}: Extensive libraries, documentation, and community support.
\end{enumerate}

\begin{table}[h]
\centering
\begin{tabular}{@{}llll@{}}
\toprule
Framework & Performance & Code Sharing & Learning Curve \\
\midrule
React Native & Near-native & 90\%+ & Medium \\
Flutter & Native & 95\%+ & High \\
Ionic & Web-based & 100\% & Low \\
Native (Swift/Kotlin) & Native & 0\% & High \\
\bottomrule
\end{tabular}
\caption{Mobile Framework Comparison}
\end{table}

\subsection{Expo SDK 52}

Expo provides a managed workflow for React Native:

\begin{enumerate}
    \item \textbf{Zero Configuration}: No Xcode or Android Studio setup required for development.
    
    \item \textbf{OTA Updates}: Push updates without app store review.
    
    \item \textbf{Expo Go}: Test on physical devices instantly via QR code.
    
    \item \textbf{EAS Build}: Cloud-based builds for production releases.
    
    \item \textbf{SDK Libraries}: Pre-configured access to device features (camera, storage, etc.).
\end{enumerate}

\subsection{React Navigation 7}

React Navigation handles screen navigation:

\begin{enumerate}
    \item \textbf{Native Feel}: Uses native navigation primitives on each platform.
    \item \textbf{Type Safety}: Full TypeScript support for route parameters.
    \item \textbf{Deep Linking}: URL-based navigation for external links.
    \item \textbf{Tab Navigation}: Bottom tab bar with custom styling.
\end{enumerate}

\begin{lstlisting}[language=JavaScript, caption=Navigation Setup]
const Tab = createBottomTabNavigator();

function AppNavigator() {
  return (
    <Tab.Navigator tabBar={(props) => <CustomTabBar {...props} />}>
      <Tab.Screen name="Home" component={HomeScreen} />
      <Tab.Screen name="Chat" component={ChatScreen} />
      <Tab.Screen name="Resources" component={ResourcesScreen} />
      <Tab.Screen name="Settings" component={SettingsScreen} />
    </Tab.Navigator>
  );
}
\end{lstlisting}

\subsection{Zustand with AsyncStorage}

Zustand provides lightweight state management:

\begin{enumerate}
    \item \textbf{Minimal Boilerplate}: Simpler than Redux for React Native.
    \item \textbf{Persistence}: AsyncStorage middleware for offline data.
    \item \textbf{TypeScript}: Excellent type inference.
\end{enumerate}

\begin{lstlisting}[language=JavaScript, caption=Persistent Auth Store]
export const useAuthStore = create<AuthState>((set) => ({
  user: null,
  accessToken: null,
  
  signIn: async (email, password) => {
    const response = await api.login(email, password);
    await AsyncStorage.setItem('@auth', JSON.stringify(response));
    set({ user: response.user, accessToken: response.access_token });
  },
}));
\end{lstlisting}

\section{Architecture Decisions}

\subsection{Lazy Login Pattern}

The app implements lazy login for accessibility:

\begin{itemize}
    \item Users can access chat without authentication
    \item Authentication is optional for data synchronization
    \item Reduces friction for users seeking immediate help
\end{itemize}

\subsection{RTL Support}

Arabic language support requires RTL layouts:

\begin{lstlisting}[language=JavaScript, caption=RTL Configuration]
import { I18nManager } from 'react-native';

// Enable RTL for Arabic
if (language === 'ar' || language === 'dz') {
  I18nManager.forceRTL(true);
}
\end{lstlisting}

\subsection{API Service Layer}

Centralized API communication:

\begin{lstlisting}[language=JavaScript, caption=API Service]
const API_BASE_URL = 'http://192.168.x.x:8000';

export async function sendChatMessage(message: string) {
  const response = await fetch(`${API_BASE_URL}/chat`, {
    method: 'POST',
    headers: { 'Content-Type': 'application/json' },
    body: JSON.stringify({ message }),
  });
  return response.json();
}
\end{lstlisting}

\section{Performance Considerations}

\begin{enumerate}
    \item \textbf{FlatList}: Virtualized lists for chat messages
    \item \textbf{Memoization}: React.memo for expensive components
    \item \textbf{Image Optimization}: Cached and resized images
    \item \textbf{Bundle Splitting}: Lazy loading for screens
\end{enumerate}

\section{Conclusion}

React Native with Expo provides the optimal solution for Amal's mobile requirements. The combination of cross-platform development, native performance, and the React ecosystem enables rapid development while maintaining quality user experience across Android and iOS.

\section{References}

\begin{enumerate}
    \item React Native Documentation: \url{https://reactnative.dev}
    \item Expo Documentation: \url{https://docs.expo.dev}
    \item React Navigation: \url{https://reactnavigation.org}
    \item Zustand: \url{https://github.com/pmndrs/zustand}
\end{enumerate}

\end{document}
